\documentclass[french]{article}
\usepackage{amssymb, amsmath} %pour les mathématiques
\usepackage{fontspec}
\usepackage{xunicode}
\usepackage[a4paper]{geometry}
\usepackage{babel}
\setmainfont{Linux Libertine O}
\begin{document}
\section{Introduction}
\section{Estimer son budget}
Commencez par faire le point sur vos entrées d'argent (salaire, aides sociales, revenus de la bourse, etc.) et de vos frais (crédit en cours, pension, ce qu'il vous faut en dehors du loyer pour vivre). Prenez votre taxe d'habitation comme estimation de celle que vous paierez et tenter une estimation de la taxe foncière.

Pour estimer correctement le budget il faut tenir compte de tous les postes :
\begin{itemize}
\item Prix du terrain ;
\item Frais d'acte ;
\item Frais d'agence ;
\item Etude de sol ;
\item Frais de raccordement Elec ;
\item Frais de raccordement Eau ;
\item Frais de raccordement Téléphone ;
\item Taxe d'aménagement ;
\end{itemize}

\subsection{Taxe d'aménagement}
Contacter l'organisme. Mais il faut une estimation de la surface de la maison et des éventuels garage et parties couvertes.

\subsection{Astuce}
Si votre commune l'a voté en conseil d'administration vous pouvez demander à être exonéré de la taxe foncière.

\section{Intermédiaire : contacter des architectes}
Fort de votre budget et de votre esquisse de plan, allez voir deux ou trois architectes pour vous assurer que votre projet est réalisable.

\section{Rechercher le terrain}
Quand un terrain semble vous intéresser vraiment, contacter la mairie pour vérifier l'état des réseaux, des écoles, des bus...

\section{Préparer le compromis}
En particulier les clauses suspensives

\end{document}