\documentclass[french]{book}
\usepackage{amssymb, amsmath} %pour les mathématiques
\usepackage{fontspec}
\usepackage{xunicode}
\usepackage[a4paper]{geometry}
\usepackage{babel}
\setmainfont{Linux Libertine O}
\title{Construire serein}
\author{Construire serein}
\date{21 août 2015}

\begin{document}
\maketitle
\newpage
\tableofcontents
% Pour la page de titre ajouter le logo suivant : un petit bonhomme stylisé (bonhomme bâton), souriant, tenant dans sa main droite à l'horizontal le logo maison du site construire-serein.fr. Le bonhomme regarde vers la maison dans sa main.

% Pour mettre une image en fond : \usepackage{wallpaper}
\chapter{Introduction}

\chapter{Se préparer}
\section{Estimer son budget}
Commencez par faire le point sur vos entrées d'argent (salaire, aides sociales, revenus de la bourse, etc.) et de vos frais (crédit en cours, pension, ce qu'il vous faut en dehors du loyer pour vivre). Prenez votre taxe d'habitation comme estimation de celle que vous paierez et tenter une estimation de la taxe foncière.

Pour estimer correctement le budget il faut tenir compte de tous les postes :
\begin{itemize}
    \item Prix du terrain ;
    \item Frais d'acte ;
    \item Frais d'agence ;
    \item Etude de sol ;
    \item Frais de raccordement Électrique ;
    \item Frais de raccordement Eau ;
    \item Frais de raccordement Téléphone ;
    \item Taxe d'aménagement ;
\end{itemize}

\subsection{Faire point sur ses revenus et ses charges}
\subsection{Déterminer sa capacité d’emprunt}
\subsection{La règle des 33\%}
\subsection{Conclusion}
Faites la somme de votre apport personnel, de votre capacité d'emprunt et des éventuelles sommes d'argent dont vous pourriez disposer et vous avez là une idée assez précise de la quantité d'argent dont vous disposez pour votre projet. 

Il s'agit d'une première estimation car le montage financier pourra augmenter cette capacité.

\section{Définir la maison de ses rêves}
Répondez au questionnaire suivant :
\begin{itemize}
    \item Combien de chambres ?
    \item Combien de salles de bain ?
    \item Cave ?
    \item ...
\end{itemize}

\section{Rencontrer des architectes}
Une fois votre budget et la maison de vos rêves en main aller voir deux ou trois architectes pour leurs présenter votre projet. Cela vous permettra, bien avant de devoir vous engager avec un architecte, de voir comment ça se passe, si le courant passe bien et surtout vous aurez une idée de ce que vous allez devoir enlever dans la maison de vos rêves pour tenir dans votre budget.

\section{Définir la zone de recherche}

\chapter{Le montage financier}
\section{L'apport personnel}
\section{Le prêt principal}
\section{Les prêts à taux zéro}
\section{Le 1\% logement}
\section{Les aides}
La mairie, le département, la région, certains organismes peuvent prêter de l'argent dans le cas de constructions neuves. Par exemple, dans le gers, il existe une aide pour clôturer son terrain avec une haie composée d'espèces locales.


\chapter{Mener à bien son projet, étape par étape}
On donne ici le planning des différentes étapes (recherche terrain..., le point de départ à partir duquel sont donnés les délai est la signature du compromis)



\section{Ce que conçoit votre architecte}
Lorsqu'on pense son projet de maison on imagine la maison, bien sûr, mais on pense aussi au jardin, à la clôture, la piscine, l'abri jardin, l'assainissement, les réseaux... Bref, on pense sa maison dans sa globalité avec son environnement. 

Pour un architecte, la vision est différente. Lui pense uniquement à la maison, le bâtiment maison. C'est une des premières causes de dépassement de budget, ne serait-ce que pour la clôture par exemple. Il faut compter 10000 euros facilement pour une clôture avec portail faisant le tour du terrain.

On a tendance à faire confiance à son architecte, en pensant qu'il voit le projet de la même manière que nous. C'est la première erreur à éviter pour réussir sa construction. En première approximation, partez de l'hypothèse que l'architecte va compter pour le budget que vous allez lui donner uniquement les éléments suivants :
\begin{itemize}
    \item le bâtît de la maison ;
    \item le crépis ou le revêtement extérieur ;
    \item les menuiseries ;
    \item les sols ;
\end{itemize}

Ce que l'architecte ne comptera pas, et vous allez être surpris et comprendre pourquoi les gens dépassent le budget initial de 20\% :
\begin{itemize}
    \item les sanitaires (robinetterie, douche, baignoire, WC, meubles salle de bain). Il comptera la pose mais pas le matériel ;
    \item les clôtures ;
    \item l'assainissement ;
    \item le raccordement aux réseaux (eau, électricité, téléphone, gaz) ;
    \item les VRD (Voirie et Réseaux Divers) ;
    \item le tour de maison (les petits gravier pour rester au sec sur tout le tour de la maison) ;
    \item le paysage du jardin (vous risquez même de vous retrouver à la livraison avec un terrain labourré par les camions !!) ;
    \item les dépendances (abri jardin par exemple) ;
    \item les consommables (ampoules par exemple et ça peut vite coûter un bras d'équiper toute une maison)
    \item la cuisine, ses meubles, son équipement ;
    \item les peintures, tapisseries, etc. ;
    \item les radiateurs ;
    \item tout ce qui est extérieur au bâti.
\end{itemize}

Et cette liste n'est pas exhaustive, il faut l'adapter à chaque projet.

Le coût au mètre carré



\section{Déterminer le prix moyen du mètre carré}
\section{Trouver la surface moyenne que vous pouvez avoir}
\section{Affinez les exigences de votre maison}
\section{Pensez aux intérieurs}
\section{Pensez à l’extérieur}
\section{Négocier son contrat avec l’architecte}
\section{Suivre le travail de l’architecte}
\section{L’étude de sol}
\section{L’assainissement}



\subsection{Taxe d'aménagement}
Contacter l'organisme. Mais il faut une estimation de la surface de la maison et des éventuels garage et parties couvertes.

\subsection{Astuce}
Si votre commune l'a voté en conseil d'administration vous pouvez demander à être exonéré de la taxe foncière.

\section{Intermédiaire : contacter des architectes}
Fort de votre budget et de votre esquisse de plan, allez voir deux ou trois architectes pour vous assurer que votre projet est réalisable.

\section{Rechercher le terrain}
Quand un terrain semble vous intéresser vraiment, contacter la mairie pour vérifier l'état des réseaux, des écoles, des bus...

\section{Préparer le compromis}
En particulier les clauses suspensives

\chapter{Trouver les fonds}
\chapter{La conception de la maison}
\chapter{Le suivi du chantier}

\end{document}